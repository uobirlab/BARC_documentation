
\section{Team members}
%from rules:  Main involved research areas in the team work
Currently, the team has eight active members - three bachelor's, one master's and three PhD students. All team members are students in the School of Computer Science, University of Birmingham. Bachelor students are getting familiar with ROS and robotics. Hence, they do not have any specific background, research interests or position in the team yet. Therefore, the members description and details are only given for the master's student and the PhD students. The final team line-up is likely to change before the competition as more members will contribute.

\subsection{Lenka Mudrova}

She is a PhD student with research interests in AI planning and scheduling. These aspects are important techniques for a robot to make decisions on when, how and what needs to be performed. Such decisions are necessary in service robotics when a robot needs to complete tasks assigned by a human. The robot needs to have a control framework that makes decisions concerning which particular task should be executed. The quality of the decisions influence the overall performance of the robot and of course, the robot's goal must satisfy as many of the requirements assigned by the humans as possible.

In the team, she has two roles. First, she is the team leader, which mainly includes representation of the team when formal communication outside of the team is necessary. Also, she makes sure that every member of the team knows what is happening, how their modules will be used within the system and what is required from them. The second role is that she is working also on the robot's subsystems. Her research interest in AI planning will be useful for creating the robot's overall behaviour. Moreover, she is working in computer vision and speech recognition. Lastly she contributes with her experience as she was involved as a team leader of the student robotics team FELaaCZech that took part in the international competition Eurobot for four years. 

%% Dunno if we should include that
%Her research is part of the EU STRANDS project \cite{strands}. "STRANDS aims to enable a robot to %achieve robust and intelligent behaviour in human environments through adaptation to, and the %exploitation of, long-term experience (at least 120 days by the end of the project).". Therefore, %the robot's control framework will exploit long-term experiences and observations of the robot's %world. 

\subsection{Marco Antonio Becerra Pedraza}

He is a PhD student in the School of Computer Science. His research interests are 3D perception, human sensing, knowledge representation and reasoning. More specifically his PhD research is about semantic mapping of human events. The topic has two main components: a) Semantic mapping, which can be conceived as an extension of the mapping problem \cite{Nuchter08_TowardsSemanticMaps}. The environment has additional spatial information that needs to be handled. Semantic maps extend the concept of maps to handle more features from the environment (e.g. structure, functionalities, events); b) Activity recognition is about using observations from the world to build representations of the ongoing actions. Later, these representations can be used to find associated structures.

Inside the team he has worked with the human sensing capabilities of the robot. He is also investigating how his PhD research can contribute to our current system. Lastly, he is a former member of the PUMAS RoboCup@Home team and as such, brings useful experience to the team.

\subsection{Manolis Chiou}

He is a PhD student with a multidisciplinary background in control engineering, robotics and AI. His PhD work addresses the problem of variable autonomy in teleoperated mobile robots. Variable autonomy refers to the different levels of autonomous capabilities that are implemented on a robot. This can be potentially useful for robots used on demanding and safety critical tasks (e.g. search and rescue, hazardous environments inspection, bomb disposal), which are currently teleoperated and could soon start to benefit from autonomous capabilities. Robots could usefully use AI control algorithms to autonomously take control of certain functions when the human operator is suffering a high workload, high cognitive load, anxiety, or other distractions and stresses. In contrast, some circumstances may still necessitate direct human control of the robot. His research aims to tackle the problem by designing control algorithms for switching between the different autonomy levels in an optimal way. 

He co-leads BARC and he shares responsibilities in administration and organization of the team with Lenka. His hands-on work on the system involves but is not limited to control, navigation, localization, data logging and Human-Robot-Interaction (HRI).




\subsection{Sean Bastable}

He is a master's student in Robotics and is expected to follow up with a PhD in computer vision. He has worked on a number of student projects related to robot vision such as the task of visual localization of a mobile robot. This involves the use of a ceiling facing omnidirectional camera and PCA in order to determine an estimate of the robot's location relative to a set of training data. This approach is particularly useful in situations when other localization methods may be unavailable or unreliable. For example, laser based localization methods may produce incorrect location estimates in crowded environments where key features of the known environment may not be visible to the robot. Recently, he has been working on object recognition using convolutional neural networks (CNNs).


As the oldest BARC member, he has solid experience with ROS and various robotic platforms. His work on Dora includes face detection and recognition, object perception and overall system integration.