\begin{figure}[!t]
\centering
\includegraphics[width=2.in]{dora_new.png}
\caption{Dora is an extended Pioneer 3D-X robot with sensors such as a laser range finder, depth camera and a laptop mounted on top.}
\label{fig:dora}
\end{figure}

Our robot Dora is created using the differential platform Pioneer 3D-X and central supporting construction for sensors and a laptop, see Fig~\ref{fig:Dora}. 
In last year, Dora was equipped with 2 laser rangefinders and 1 depth camera, see overview in Tab.~\ref{tab:c_sensors}, which were used to ensure safe navigation 
and people detection. 
This year, we are going to do significant improvements of her hardware, see Tab.~\ref{tab.n_sensors}, providing more support for:
\begin{itemize}
\item \textit{human robot interaction} - Dora's speech understanding and speech reproduction were strongly limited to the used laptop last year. 
We are going to extend Dora by mounting a microphone and a speaker on her. Additionally, we are going to improve her apperiance to be more user-friendly. 
\item \textit{safe and robust navigation} - We are investigating usage of Pioneer's sonar to detect objects shorter than height when laser rangefinders are mounted. 
Moreover, Pioneer's bumpers will be used to stop the robot when an object is hit. 
\item \textit{object recognition} - another depth camera with short range is added to allow detection of small household objects.
\end{itemize}
  

\begin{table}
\begin{tabular}{lll}
\hline
sensor & parameters & usage\\
\hline
\hline

%the laser rangefinder SICK TIM551-2050001 which has range up to 10 m, 
%270$^\circ$
%field of view with resolution 1$^\circ$ and low sensitivy to ambient light. 

%Hokuyo laser scanner

%pan-tilt unit
\end{tabular}
\caption{\label{tab:c_sensors}}
\end{table}

%\begin{table}
%\begin{tabular}

%\end{tabular}
%\caption{\label{tab:n_sensors}}
%\end{table}


