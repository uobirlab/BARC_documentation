

\documentclass[conference]{IEEEtran}


\usepackage{cite}




% *** GRAPHICS RELATED PACKAGES ***
%
\ifCLASSINFOpdf
  \usepackage[pdftex]{graphicx}
  % declare the path(s) where your graphic files are
  % \graphicspath{{../pdf/}{../jpeg/}}
  % and their extensions so you won't have to specify these with
  % every instance of \includegraphics
  % \DeclareGraphicsExtensions{.pdf,.jpeg,.png}
\else
  % or other class option (dvipsone, dvipdf, if not using dvips). graphicx
  % will default to the driver specified in the system graphics.cfg if no
  % driver is specified.
  % \usepackage[dvips]{graphicx}
  % declare the path(s) where your graphic files are
  % \graphicspath{{../eps/}}
  % and their extensions so you won't have to specify these with
  % every instance of \includegraphics
  % \DeclareGraphicsExtensions{.eps}
\fi


% *** MATH PACKAGES ***
%
%\usepackage[cmex10]{amsmath}
% A popular package from the American Mathematical Society that provides
% many useful and powerful commands for dealing with mathematics. If using
% it, be sure to load this package with the cmex10 option to ensure that
% only type 1 fonts will utilized at all point sizes. Without this option,
% it is possible that some math symbols, particularly those within
% footnotes, will be rendered in bitmap form which will result in a
% document that can not be IEEE Xplore compliant!
%
% Also, note that the amsmath package sets \interdisplaylinepenalty to 10000
% thus preventing page breaks from occurring within multiline equations. Use:
%\interdisplaylinepenalty=2500
% after loading amsmath to restore such page breaks as IEEEtran.cls normally
% does. amsmath.sty is already installed on most LaTeX systems. The latest
% version and documentation can be obtained at:
% http://www.ctan.org/tex-archive/macros/latex/required/amslatex/math/



% *** SPECIALIZED LIST PACKAGES ***
%
%\usepackage{algorithmic}
% algorithmic.sty was written by Peter Williams and Rogerio Brito.
% This package provides an algorithmic environment fo describing algorithms.
% You can use the algorithmic environment in-text or within a figure
% environment to provide for a floating algorithm. Do NOT use the algorithm
% floating environment provided by algorithm.sty (by the same authors) or
% algorithm2e.sty (by Christophe Fiorio) as IEEE does not use dedicated
% algorithm float types and packages that provide these will not provide
% correct IEEE style captions. The latest version and documentation of
% algorithmic.sty can be obtained at:
% http://www.ctan.org/tex-archive/macros/latex/contrib/algorithms/
% There is also a support site at:
% http://algorithms.berlios.de/index.html
% Also of interest may be the (relatively newer and more customizable)
% algorithmicx.sty package by Szasz Janos:
% http://www.ctan.org/tex-archive/macros/latex/contrib/algorithmicx/



% *** ALIGNMENT PACKAGES ***
%
%\usepackage{array}
% Frank Mittelbach's and David Carlisle's array.sty patches and improves
% the standard LaTeX2e array and tabular environments to provide better
% appearance and additional user controls. As the default LaTeX2e table
% generation code is lacking to the point of almost being broken with
% respect to the quality of the end results, all users are strongly
% advised to use an enhanced (at the very least that provided by array.sty)
% set of table tools. array.sty is already installed on most systems. The
% latest version and documentation can be obtained at:
% http://www.ctan.org/tex-archive/macros/latex/required/tools/


%\usepackage{eqparbox}
% Also of notable interest is Scott Pakin's eqparbox package for creating
% (automatically sized) equal width boxes - aka "natural width parboxes".
% Available at:
% http://www.ctan.org/tex-archive/macros/latex/contrib/eqparbox/


% *** SUBFIGURE PACKAGES ***
%\usepackage[tight,footnotesize]{subfigure}
% subfigure.sty was written by Steven Douglas Cochran. This package makes it
% easy to put subfigures in your figures. e.g., "Figure 1a and 1b". For IEEE
% work, it is a good idea to load it with the tight package option to reduce
% the amount of white space around the subfigures. subfigure.sty is already
% installed on most LaTeX systems. The latest version and documentation can
% be obtained at:
% http://www.ctan.org/tex-archive/obsolete/macros/latex/contrib/subfigure/
% subfigure.sty has been superceeded by subfig.sty.



%\usepackage[caption=false]{caption}
%\usepackage[font=footnotesize]{subfig}
% subfig.sty, also written by Steven Douglas Cochran, is the modern
% replacement for subfigure.sty. However, subfig.sty requires and
% automatically loads Axel Sommerfeldt's caption.sty which will override
% IEEEtran.cls handling of captions and this will result in nonIEEE style
% figure/table captions. To prevent this problem, be sure and preload
% caption.sty with its "caption=false" package option. This is will preserve
% IEEEtran.cls handing of captions. Version 1.3 (2005/06/28) and later 
% (recommended due to many improvements over 1.2) of subfig.sty supports
% the caption=false option directly:
%\usepackage[caption=false,font=footnotesize]{subfig}
%
% The latest version and documentation can be obtained at:
% http://www.ctan.org/tex-archive/macros/latex/contrib/subfig/
% The latest version and documentation of caption.sty can be obtained at:
% http://www.ctan.org/tex-archive/macros/latex/contrib/caption/



% *** FLOAT PACKAGES ***
%
%\usepackage{fixltx2e}
% fixltx2e, the successor to the earlier fix2col.sty, was written by
% Frank Mittelbach and David Carlisle. This package corrects a few problems
% in the LaTeX2e kernel, the most notable of which is that in current
% LaTeX2e releases, the ordering of single and double column floats is not
% guaranteed to be preserved. Thus, an unpatched LaTeX2e can allow a
% single column figure to be placed prior to an earlier double column
% figure. The latest version and documentation can be found at:
% http://www.ctan.org/tex-archive/macros/latex/base/


%\usepackage{stfloats}
% stfloats.sty was written by Sigitas Tolusis. This package gives LaTeX2e
% the ability to do double column floats at the bottom of the page as well
% as the top. (e.g., "\begin{figure*}[!b]" is not normally possible in
% LaTeX2e). It also provides a command:
%\fnbelowfloat
% to enable the placement of footnotes below bottom floats (the standard
% LaTeX2e kernel puts them above bottom floats). This is an invasive package
% which rewrites many portions of the LaTeX2e float routines. It may not work
% with other packages that modify the LaTeX2e float routines. The latest
% version and documentation can be obtained at:
% http://www.ctan.org/tex-archive/macros/latex/contrib/sttools/
% Documentation is contained in the stfloats.sty comments as well as in the
% presfull.pdf file. Do not use the stfloats baselinefloat ability as IEEE
% does not allow \baselineskip to stretch. Authors submitting work to the
% IEEE should note that IEEE rarely uses double column equations and
% that authors should try to avoid such use. Do not be tempted to use the
% cuted.sty or midfloat.sty packages (also by Sigitas Tolusis) as IEEE does
% not format its papers in such ways.



\usepackage{url}
% \url{my_url_here}.

\usepackage[utf8]{inputenc}

% correct bad hyphenation here
\hyphenation{op-tical net-works semi-conduc-tor}


\begin{document}

% can use linebreaks \\ within to get better formatting as desired
\title{Birmingham Autonomous Robotic Club (BARC)\\
\LARGE Motto: \textit{Learning by doing}}


\author{\IEEEauthorblockN{Members: Lenka Mudrova, Manolis Chiou, Marco Becerra, \\Sean Bastable, Joshua Smith, Edoardo Bacci, Tanya Daskova and Anya Hristova}
\IEEEauthorblockA{
Email: robot-club@cs.bham.ac.uk\\
Affiliation: School of Computer Science, University of Birmingham, UK\\
Website: \textit{http://lenka-robotics.eu/index.php/robotic-club}}
}


% make the title area
\maketitle

\begin{abstract}

Birmingham Autonomous Robot Club (BARC) connects students from the University of Birmingham, with a strong interest in robotic applications and competitions. 
This paper is part of our qualification for the RoCKIn@Home 2015 competition. 
Therefore, it overviews our robot Dora and the developed software structure based on ROS middleware.

It overviews how this challenge relates to our interests and experiences and how we can achieve high reusability of our system by integrating different subsystems from other projects. Moreover, team members, their experiences and research interests are described in detail. 

Finally, the conclusion summarises our motivation and relevance for this competition.


\end{abstract}

\IEEEpeerreviewmaketitle



\section{Introduction}

Birmingham Autonomous Robotic Club (BARC) was established in 2011 in the School of Computer Science at the University of Birmingham. The main purpose was to provide an extra opportunity for students to gain additional knowledge about robotics and to work on real robotic platforms and projects. Several students involved in the team contributed to projects which were mainly used to promote robotics during the school's open days. For example a robotic waitress which accepted orders for drinks and deliver them.

In 2014, the structure of BARC was changed in order to incorporate the lessons learned and to allow the team to take part in robotics competitions. We participated successfully in Sick Days 2014 \cite{sick} and RoCKIn@Home 2014. In the latter we won two of the competition challenges. This year, we would like to participate again in RoCKIn@Home as we have still many things to learn, improve and most importantly contribute. 

The team has the support of the Intelligent Robotics Lab \cite{irlab} in the School of Computer Science. The lab conducts research and has expertise in a variety of fields including but not limited to computer vision, manipulation, planning, architectures, reasoning and mobile robots. Furthermore, the lab has strong links with the industry. 
We are also starting to cooperate with other departments within our university, such as electrical and mechanical engineering. We hope to create more interdisciplinary team than we had last year.

This paper overviews our system from last year and analyses the drawbacks of our solutions. Moreover, it extensively discusses what we are currently working on. 


\begin{figure}[!t]
\centering
\includegraphics[width=0.6\columnwidth]{logo_barc.png}
\caption{Our logo represents our university campus with famous clock tower Old Joe.}
\label{fig:dora}
\end{figure}


\section{\label{sec:hardware}Hardware}
%from rules: Description of the hardware, including an image of the robot(s)
\input hardware.tex

\section{\label{sec:software}Software architecture}

\input software.tex

\input plan.tex

\input evaluation.tex

\input team_members.tex




% An example of a floating figure using the graphicx package.
% Note that \label must occur AFTER (or within) \caption.
% For figures, \caption should occur after the \includegraphics.
% Note that IEEEtran v1.7 and later has special internal code that
% is designed to preserve the operation of \label within \caption
% even when the captionsoff option is in effect. However, because
% of issues like this, it may be the safest practice to put all your
% \label just after \caption rather than within \caption{}.
%
% Reminder: the "draftcls" or "draftclsnofoot", not "draft", class
% option should be used if it is desired that the figures are to be
% displayed while in draft mode.
%
%\begin{figure}[!t]
%\centering
%\includegraphics[width=2.5in]{myfigure}
% where an .eps filename suffix will be assumed under latex, 
% and a .pdf suffix will be assumed for pdflatex; or what has been declared
% via \DeclareGraphicsExtensions.
%\caption{Simulation Results}
%\label{fig_sim}
%\end{figure}

% Note that IEEE typically puts floats only at the top, even when this
% results in a large percentage of a column being occupied by floats.


% An example of a double column floating figure using two subfigures.
% (The subfig.sty package must be loaded for this to work.)
% The subfigure \label commands are set within each subfloat command, the
% \label for the overall figure must come after \caption.
% \hfil must be used as a separator to get equal spacing.
% The subfigure.sty package works much the same way, except \subfigure is
% used instead of \subfloat.
%
%\begin{figure*}[!t]
%\centerline{\subfloat[Case I]\includegraphics[width=2.5in]{subfigcase1}%
%\label{fig_first_case}}
%\hfil
%\subfloat[Case II]{\includegraphics[width=2.5in]{subfigcase2}%
%\label{fig_second_case}}}
%\caption{Simulation results}
%\label{fig_sim}
%\end{figure*}
%
% Note that often IEEE papers with subfigures do not employ subfigure
% captions (using the optional argument to \subfloat), but instead will
% reference/describe all of them (a), (b), etc., within the main caption.


% An example of a floating table. Note that, for IEEE style tables, the 
% \caption command should come BEFORE the table. Table text will default to
% \footnotesize as IEEE normally uses this smaller font for tables.
% The \label must come after \caption as always.
%
%\begin{table}[!t]
%% increase table row spacing, adjust to taste
%\renewcommand{\arraystretch}{1.3}
% if using array.sty, it might be a good idea to tweak the value of
% \extrarowheight as needed to properly center the text within the cells
%\caption{An Example of a Table}
%\label{table_example}
%\centering
%% Some packages, such as MDW tools, offer better commands for making tables
%% than the plain LaTeX2e tabular which is used here.
%\begin{tabular}{|c||c|}
%\hline
%One & Two\\
%\hline
%Three & Four\\
%\hline
%\end{tabular}
%\end{table}


% Note that IEEE does not put floats in the very first column - or typically
% anywhere on the first page for that matter. Also, in-text middle ("here")
% positioning is not used. Most IEEE journals/conferences use top floats
% exclusively. Note that, LaTeX2e, unlike IEEE journals/conferences, places
% footnotes above bottom floats. This can be corrected via the \fnbelowfloat
% command of the stfloats package.

\section{Conclusion}
%TODO proof reading
%from rules: Applicability and relevance to domestic/industrial robotics [@Home / @Work]
%Lenka
Our team has strong relevance to the domestic service robotics, as
Lenka's, Marco's and Manolis's research interests involved cooperation with
humans. We would like to use our expertises to combine the state of the art
in AI techniques with our own contributions. As a result, a robotic system
will be created with high reusability, as we use ROS middleware.

We would like to achieve high robustness of our robotic system that it can perform tasks repeatedly. In order of this our goal, we would like to put attention to verification and evaluation of our system. We expect that our robot Dora will be able to perform completely
the \textit{Welcoming visitors} task, \textit{Getting to know my home} task and \textit{Navigation} task. In order of fulfil these tasks, we
plan to merge techniques for face and an uniform detection, speech
recognition and synthesis, navigation, people detection, HRI behaviour that
will ensure that a person is following the robot, door detection, following
a person and gesture recognition.

We would like to perform other tasks as well, but it depends on our human resources.
We plan to cooperate more with first year bachelor's students in
order to introduce them to robotics and provide them with some interesting
challenges, which can be used during competition. We believe that
participating in the RoCKIn@Home challenge will bring a lot of experience
not only to the young students, but also to us. As a result, we are
expecting to obtain more knowledge in the ongoing research in many fields
of robotics.

%All team members have research interests in areas where a robot must cooperate with people in order to provide the best performance. To summarize, Lenka is working on the STRANDS project, which aims to develop robots capable of helping elderly people with every day activities. Marco is interested in activity recognition which can significantly improve the cooperation of robots and humans. Manolis is doing research on flexible robotic control via cooperation between a human and an autonomous robot, especially in environments which are difficult for a robot. Sean did his bachelor's project on visual robot localisation in human crowds. Therefore, we believe that our work is strongly relevant to the RoCKIn@Home challenge.

%Moreover, if our team participates in this challenge, it will be an interesting challenge especially to newly incoming members, as we are expecting that more first year bachelor's students will join the team this summer. 



% conference papers do not normally have an appendix


% use section* for acknowledgement
%\section*{Acknowledgment}



% trigger a \newpage just before the given reference
% number - used to balance the columns on the last page
% adjust value as needed - may need to be readjusted if
% the document is modified later
%\IEEEtriggeratref{8}
% The "triggered" command can be changed if desired:
%\IEEEtriggercmd{\enlargethispage{-5in}}

% references section

% can use a bibliography generated by BibTeX as a .bbl file
% BibTeX documentation can be easily obtained at:
% http://www.ctan.org/tex-archive/biblio/bibtex/contrib/doc/
% The IEEEtran BibTeX style support page is at:
% http://www.michaelshell.org/tex/ieeetran/bibtex/


\bibliographystyle{IEEEtran}
\bibliography{./ref}
%
% <OR> manually copy in the resultant .bbl file
% set second argument of \begin to the number of references
% (used to reserve space for the reference number labels box)
%\begin{thebibliography}{1}

%\bibitem{IEEEhowto:kopka}
%H.~Kopka and P.~W. Daly, \emph{A Guide to \LaTeX}, 3rd~ed.\hskip 1em plus
%0.5em minus 0.4em\relax Harlow, England: Addison-Wesley, 1999.

%\end{thebibliography}




% that's all folks
\end{document}


